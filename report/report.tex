\documentclass[11pt,a4paper]{article}

% Packages
\usepackage[utf8]{inputenc}
\usepackage[T1]{fontenc}
\usepackage[english]{babel}
\usepackage{geometry}
\usepackage{setspace}
\usepackage{graphicx}
\usepackage{amsmath}
\usepackage{array}
\usepackage[colorlinks=true, linkcolor=black, urlcolor=blue, citecolor=black]{hyperref}
\usepackage{fancyhdr}
\usepackage{caption}
\usepackage{booktabs}
\usepackage{ragged2e}
\usepackage{enumitem}
\usepackage{amsmath, amssymb, booktabs}
\usepackage{siunitx}
\usepackage{multicol}
\usepackage{rotating}
\usepackage{tikz}
\usepackage{float}


% Arial-like font
\renewcommand{\familydefault}{\sfdefault}

% Page geometry
\geometry{
  a4paper,
  left=2.5cm,
  right=3.0cm,
  top=2.5cm,
  bottom=2.0cm
}

% Font and spacing
\onehalfspacing
\renewcommand{\familydefault}{\sfdefault}

% Title page
\begin{document}

\begin{titlepage}
\centering
% \vspace*{2cm}
  
\includegraphics[width=0.7\textwidth]{imgs/USB.png}\\[1cm]

\vspace{1.5cm}

\textbf{\Large University of South Bohemia} \\
\textbf{\large Faculty of Science} \\
\large Artificial Intelligence and Data Science, M.Sc. \\[2.5cm]

{\Large \textsc{Designing a Bloom Filter for a Target False Positive Rate}}\\[0.5cm]

Submitted for the course \textbf{Information Theory} \\
\textbf{Professor:} Kaštovský Jan prof. RNDr. Ph.D. \\[3cm]

\begin{flushleft}
\begin{tabular}{ll}
Submitted by: & María Isabel Sánchez-O'Mullony Martínez \\
Student ID: & B24763 \\
Date: & \today \\
\end{tabular}
\end{flushleft}

\vspace{3cm}

\begin{flushright}
\begin{tabular}{ll}
Supervisor: & Kaštovský Jan prof. RNDr. Ph.D. \\
\end{tabular}
\end{flushright}

\end{titlepage}


% ===============================
% Declaration
% ===============================
\section*{Declaration}
I declare that I have written this report by myself and have only used the sources and aids mentioned, and that I have 
marked direct and indirect citations as such. This report has not been submitted prior for any other examination.

I agree that the results of this study work / report may be used free of charge for research and lecturing purposes.

\newpage

% Table of Contents
\tableofcontents
\newpage

% List of Abbreviations
\section*{List of Abbreviations}
\begin{tabular}{ll}
    FPR & False-Positive Rate \\
\end{tabular}

\newpage

% Main Report Sections
% ===============================
% Introduction
% ===============================
\section{Introduction}\label{sec:Introduction}
A Bloom filter is a space-efficient probabilistic data structure that that supports set membership queries, testing if
an element is a member of a set\cite{tarkoma2011theory}. The structure offers some interesting properties: It offers
a compact probabilistic way to represent a set that can result in false positives (claiming an element to be part of 
the set when it is not), but never in false negatives (reporting an inserted element to be absent from the set). A
Bloom filer of a fixed size can represent a set with an arbitrary large number of elements. Adding an element never
fails, however, we need to take into account the increasing probability of false positive, and that deleting an 
element from filter is not possible\cite{tarkoma2011theory, geeksforgeeks_bloom_filters}.

The objective of this project is to design and implement a Bloom filter that automatically chooses the optimal size 
(m) and number of hash functions (k) to achieve a target false-positive rate (FPR), given an expected number of 
elements n.


\newpage

\cleardoublepage
\addcontentsline{toc}{section}{References}
\bibliographystyle{IEEEtran}
\bibliography{references}

\end{document}
